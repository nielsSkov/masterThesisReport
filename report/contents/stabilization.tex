\chapter{Stabilization}
In this section the idea is to stabilize the pendulum in the unstable equilibrium. Ultimately this controller should be able to take over from the swing-up controller when some minimum catch angle is reached.\\
A sliding mode control strategy is employed to accomplish these goals. The design is based on \cite{HKKhalil}.\\
Firstly, the model of the system, from \autoref{eq:nonlinearStateSpace}, is considered in following form,
\begin{align}
  \begin{bmatrix}
    \dot{x}_1 \\
    \dot{x}_2 \\
    \dot{x}_3 \\
    \dot{x}_4
  \end{bmatrix}
  &=
  \underbrace{
    \begin{bmatrix}
      x_3 \\
      x_4 \\
      \vec{M}^{-1}(x_1) ( - \vec{C}(x_1,x_3) - \vec{B}(x_3,x_4) - \vec{G}(x_1) )
    \end{bmatrix}
  }_{\vec{f}(\vec{x})}
  +
  \underbrace{ 
    \begin{bmatrix}
      0 \\
      0 \\
      \vec{M}^{-1}(x_1) \vec{F} 
    \end{bmatrix}
  }_{\vec{g}(\vec{x}) u}
  \label{eq:nonlinearStateSpace2} \ \ \ ,
\end{align}
where
\begin{align}
  \vec{M}^{-1}
  &=
  \begin{bmatrix}
    \frac{(M + m)}{l^2 m ( M + m - m \cos^2 x_1 )}  &  \frac{\cos x_1}{l (M + m - m \cos^2 x_1)} \\
    \frac{\cos x_1}{l (M + m - m \cos^2 x_1)}       &  \frac{1}{M + m - m \cos^2 x_1}
  \end{bmatrix}  \ \ \ ,
\end{align}
%
and states $ [\ x_1\ \ x_2\ \ x_3\ \ x_4\ ]^\mathrm{T} = [\ \theta\ \ x\ \ \dot{\theta}\ \ \dot{x}\ ]^\mathrm{T} $ and input vector $\vec{F} = [\ 0 \ \ u \ ]^\mathrm{T}$ as before.

In \autoref{eq:nonlinearStateSpace2} the input, $u$, appear in two of the four state equations. To design a sliding mode controller for the system, it is transformed into \textit{regular form}, 
\begin{align}
\dot{\vec{\eta}} &=  \vec{f_a}(\vec{\eta},\xi) \nonumber   \\
\dot{\xi}        &=  f_b(\vec{\eta},\xi) + g_b(\vec{\eta},\xi) u    \ \ \ ,
\label{eq:regularForm}
\end{align}
%
where the input only appears on one state equation.
The transform is then given by,
\begin{align}
  \vec{T}(\vec{x}) &=
  \begin{bmatrix}
    \vec{\eta} \\
    \xi
  \end{bmatrix}
  \ \Rightarrow \ 
  \frac{\partial}{\partial t}\vec{T}(\vec{x})
  =
  \begin{bmatrix}
    \dot{\vec{\eta}} \\
    \dot{\xi}
  \end{bmatrix}
  \ \Rightarrow \ 
  \frac{\partial}{\partial t}\vec{T}(\vec{x})
  =
  \begin{bmatrix}
    \vec{f_a}(\vec{\eta},\xi)         \\
    f_b(\vec{\eta},\xi) + g_b(\vec{\eta},\xi) u
  \end{bmatrix}
   \ \ \ ,
  \label{eq:transformAndDerivative}
\end{align}
further,
\begin{align}
  \frac{\partial \vec{T}}{\partial t} &= \frac{\partial \vec{T}}{\partial \vec{x}}  \dot{\vec{x}} \\
  \begin{bmatrix}
    \vec{f_a}(\vec{\eta},\xi)         \\
    f_b(\vec{\eta},\xi) + g_b(\vec{\eta},\xi) u
  \end{bmatrix}
  &=
  \frac{\partial \vec{T}}{\partial \vec{x}} \vec{f}(\vec{x}) + \frac{\partial \vec{T}}{\partial \vec{x}} \vec{g}(\vec{x}) u
  \ \ \ ,
  \label{eq:transformDerivative}
\end{align}
such that,
\begin{align}
  \frac{\partial \vec{T}}{\partial \vec{x}} \vec{f}(\vec{x})    
  &= 
  \begin{bmatrix}
    \vec{f_a}(\vec{\eta},\xi)       \\
    f_b(\vec{\eta},\xi)
  \end{bmatrix}  \ \ \ \ ,\ \ \ \ 
  \frac{\partial \vec{T}}{\partial \vec{x}} \vec{g}(\vec{x})
  = 
  \begin{bmatrix}
    \vec{0}        \\
    g_b(\vec{\eta},\xi)
  \end{bmatrix}
  \ \ \ .
  \label{eq:transformXDerivative}
\end{align}
\autoref{eq:transformXDerivative} results in the following four equations,
\begin{align}
    \frac{\partial \eta_1}{\partial x_3} g_3 + \frac{\partial \eta_1}{\partial x_4} g_4 &= 0                    \ \ \ \ ,\ \ \ \ % \label{eq:chooseEta1}  \\
    \frac{\partial \eta_2}{\partial x_3} g_3 + \frac{\partial \eta_2}{\partial x_4} g_4 = 0         \nonumber   \\ %\label{eq:chooseEta1Eta2}  \\
    \frac{\partial \eta_3}{\partial x_3} g_3 + \frac{\partial \eta_3}{\partial x_4} g_4 &= 0                    \ \ \ \ ,\ \ \ \ %\label{eq:chooseEta3}  \\
    \frac{\partial \xi   }{\partial x_3} g_3 + \frac{\partial \xi   }{\partial x_4} g_4 = g_b(\vec{\eta},\xi)  \label{eq:chooseEta1Eta2Eta3Xi} 
\ \ \ ,
\end{align}
where,
\begin{align}
  \begin{bmatrix}
    g_3  \\
    g_4
  \end{bmatrix} u = \vec{M}^{-1}(x_1) 
    \begin{bmatrix}
      0  \\
      u
    \end{bmatrix} \ \ \ \Rightarrow \ \ \ \ 
  \begin{cases}
    g_3 = \frac{\cos x_1}{l (M + m - m \cos^2 x1)}\\
    g_4 = \frac{1}{M + m - m \cos^2 x_1 }
  \end{cases}
  \label{eq:g_3_and_4} 
\end{align}
%
The following choice of coordinates to satisfy \autoref{eq:chooseEta1Eta2Eta3Xi} without loss of rank in $\vec{T}$, is based on the transform used for input-output linearization in \cite{HKKhalil}.\\
Choosing output, $h(x) = \theta$ or $h(x) = x$, both results in the relative degree, $\rho = 2$, since the output appears on the second derivatives,
\begin{align}
  \ddot{\theta} &= \dot{x}_3 = f_3 + g_3 u
  \label{eq:thetaRelativeDeg} \\
  \ddot{x} &= \dot{x}_4 = f_4 + g_4 u \ \ \ .
  \label{eq:xRelativeDeg} 
\end{align}
The suggested transform is then,
\begin{align}
  \vec{T}(\vec{x})
  =
  \begin{bmatrix}
    \phi_1(\vec{x})         \\
    \vdots                  \\
    \phi_{n-\rho}(\vec{x})  \\
    h(\vec{x})              \\
    L_f h(\vec{x})          \\
    \vdots                  \\
    L_f^{\rho-1} h(\vec{x})
  \end{bmatrix}
  \ \ \Rightarrow \ \ 
  \begin{bmatrix}
  \eta_1   \\
  \eta_2   \\
  \eta_3   \\
  \xi
  \end{bmatrix}
  =
  \begin{bmatrix}
  \phi_1(\vec{x})   \\
  \phi_2(\vec{x})   \\
  h(\vec{x})        \\
  L_f h(\vec{x})
  \end{bmatrix} \ \ \ ,
  \label{eq:transformPhi} 
\end{align}
%
where $L_f h(\vec{x})$ is the \textit{Lie derivative} of $h(\vec{x})$ along $f(\vec{x})$. This results in two possible transforms, 
\begin{align}
h = \theta \ \ \Rightarrow \ \ 
  \vec{T}_1 =
  \begin{bmatrix}
  \eta_1  \\
  \eta_2  \\
  \eta_3  \\
  \xi
  \end{bmatrix}
  =
  \begin{bmatrix}
  \phi_1  \\
  \phi_2  \\
  x_1     \\
  x_3
  \end{bmatrix} \ \ \ \mathrm{and}\ \ h = x \ \ \Rightarrow \ \
  \vec{T}_2 =
  \begin{bmatrix}
  \eta_1   \\
  \eta_2   \\
  \eta_3   \\
  \xi
  \end{bmatrix}
  =
  \begin{bmatrix}
  \phi_1  \\
  \phi_2  \\
  x_2     \\
  x_4
  \end{bmatrix} \ \ \ ,
\end{align}
leaving $\phi_1$ and $\phi_2$ to be determined. This is done by satisfying,
\begin{align}
\frac{\partial \eta_1}{\partial x_3} g_3 + \frac{\partial \eta_1}{\partial x_4} g_4 &= 0       \label{eq:chooseEta1}  \\
\frac{\partial \eta_2}{\partial x_3} g_3 + \frac{\partial \eta_2}{\partial x_4} g_4 &= 0       \label{eq:chooseEta2}  
\ \ \ .
\end{align}
from \autoref{eq:chooseEta1Eta2Eta3Xi}. For $\vec{T}_1$ the choice $\phi_1 = x_2$ satisfies \autoref{eq:chooseEta1} with no loss of rank in the transform. Conversely for $\vec{T}_2$ the choice $\phi_1 = x_1$ satisfies \autoref{eq:chooseEta1} again with no loss of rank. This leaves $\phi_2$ which, for both transforms, is determined by finding a solution to \autoref{eq:chooseEta2},
\begin{align}
 \frac{\partial \eta_2}{\partial x_3} \frac{\cos x_1}{l (M + m - m \cos^2 x1)} + \frac{\partial \eta_2}{\partial x_4} \frac{1}{M + m - m \cos^2 x_1 } = 0 \ \ \ ,
\end{align}
choosing,
\begin{align}
\frac{\partial \eta_2}{\partial x_4} = \frac{\cos x_1}{l}  \ \ , \ \ \ \frac{\partial \eta_2}{\partial x_3}  = -1 \ \ \ ,
\end{align}
such that,
\begin{align}
\eta_2 =  \frac{\cos x_1}{l} x_4 - x_3 \ \ \ .
\end{align}
%
This results in the following two transform candidates,
\begin{align}
%\begin{bmatrix}
%\eta_1   \\
%\eta_2   \\
%\eta_3   \\
%\xi
%\end{bmatrix} \left.\rule{0cm}{1.7cm}\right\vert\rule{0cm}{1.7cm}_{\substack{\rule{0cm}{1.28cm}\\ h=x_1 }}
\vec{T}_1 =
\begin{bmatrix}
x_2   \\
\frac{\cos x_1}{l} x_4 - x_3  \\
x_1   \\
x_3
\end{bmatrix} \ \ \ \ , \ \ \ \
%\begin{bmatrix}
%\eta_1   \\
%\eta_2   \\
%\eta_3   \\
%\xi
%\end{bmatrix} \left.\rule{0cm}{1.7cm}\right\vert\rule{0cm}{1.7cm}_{\substack{\rule{0cm}{1.28cm}\\ h=x_2 }}
\vec{T}_2
=
\begin{bmatrix}
x_1   \\
\frac{\cos x_1}{l} x_4 - x_3   \\
x_2   \\
x_4
\end{bmatrix} \ \ \ .
\end{align}
%
To choose a transform the following is considered.\\
A continuously differentiable map, $\vec{T}$, with a continuously differentiable inverse, $\vec{T}^{-1}$, is known as a diffeomorphism. Further, $\vec{T}$ is a global diffeomorphism iff its Jacobian is nonsingular for all $\vec{x} \in \mathbb{R}^n$ and $\lim_{||\vec{x}||\rightarrow \infty}||\vec{T}(\vec{x})|| = \infty$ , \cite{HKKhalil}.\\
The Jacobian of each transform is computed,
\begin{align}
\vec{J}_1 = \frac{\partial \vec{T}_1(\vec{x})}{\partial \vec{x}}
&=
\begin{bmatrix}
  0                       & 1 &  0 & 0                  \\
  -\frac{\sin x_1}{l} x_4 & 0 & -1 & \frac{\cos x_1}{l} \\
  1                       & 0 &  0 & 0                  \\
  0                       & 0 &  1 & 0
\end{bmatrix}  \label{eq:transform_h_x1} \\
\vec{J}_2 = \frac{\partial \vec{T}_2(\vec{x})}{\partial \vec{x}}
&=
\begin{bmatrix}
  1                       & 0 &  0 & 0                  \\
  -\frac{\sin x_1}{l} x_4 & 0 & -1 & \frac{\cos x_1}{l} \\
  0                       & 1 &  0 & 0                  \\
  0                       & 0 &  0 & 1
\end{bmatrix} \ \ \ . \label{eq:transform_h_x2}
\end{align}
To check for singularity the determinant is found for the two Jacobian matrices. In the case of \autoref{eq:transform_h_x1}, $\det(\vec{J}_1) = -\cos x_1/l$, so if $\cos x_1 = 0$ the Jacobian becomes singular. This only happens when the pendulum is horizontal which is outside the operating range of a stabilizing controller. However, for \autoref{eq:transform_h_x2}, $\det(\vec{J}_2) = 1$, and so, the Jacobian is nonsingular for all $\vec{x} \in \mathbb{R}^4$. Further, $\lim_{||\vec{x}||\rightarrow \infty}||\vec{T}(\vec{x})|| = \infty$ so,
\begin{align}
  \vec{T}
  = 
  \begin{bmatrix}
  \eta_1   \\
  \eta_2   \\
  \eta_3   \\
  \xi
  \end{bmatrix}
  =
  \begin{bmatrix}
  x_1   \\
  \frac{\cos x_1}{l} x_4 - x_3  \\
  x_2   \\
  x_4
  \end{bmatrix} \ \ \ , \label{eq:transform}
\end{align}
is a global diffeomorphism and therefore chosen as the final system transform, with the inverse given by,
\begin{align}
  \vec{T}^{-1} = 
  \begin{bmatrix}
    x_1  \\
    x_2  \\
    x_3  \\
    x_4
  \end{bmatrix}
  =
  \begin{bmatrix}
  \eta_1   \\
  \eta_3   \\
  \frac{\cos \eta_1}{l} \xi - \eta_2  \\
  \xi
  \end{bmatrix} \ \ \ .
  \label{eq:inverseTransform}
\end{align}
%
The derivative of the transform, \autoref{eq:transform}, along the trajectories of the system is,
\begin{align}
  \begin{bmatrix}
    \dot{\eta}_1   \\
    \dot{\eta}_2   \\
    \dot{\eta}_3   \\
    \dot{\xi}
  \end{bmatrix}
  &=
  \begin{bmatrix}
    \dot{x}_1   \\
    \frac{-\sin x_1}{l}\dot{x}_1 x_4 + \frac{\cos x_1}{l} \dot{x}_4 - \dot{x}_3  \\
    \dot{x}_2   \\
    \dot{x}_4
  \end{bmatrix} \label{eq:transform_dt} \\
  %
  %
  %
  \begin{bmatrix}
  \dot{\eta}_1   \\
  \dot{\eta}_2   \\
  \dot{\eta}_3   \\
  \dot{\xi}
  \end{bmatrix} 
  &=
  \begin{bmatrix}
    x_3    \\
    \frac{-\sin x_1}{l} x_3 x_4 + \frac{\cos x_1}{l} f_4(\vec{x}) + \frac{\cos x_1}{l} g_4(\vec{x}) u  - f_3(\vec{x}) - g_3(\vec{x}) u \\
    x_4    \\ 
    f_4(\vec{x}) + g_4(\vec{x}) u 
  \end{bmatrix} \ \ \ , \label{eq:transform_dt_alongTraj}
\end{align}
%
from which the regular form is obtained by rearranging and using the inverse transform,
\begin{align}
\begin{bmatrix}
  \dot{\eta}_1   \\
  \dot{\eta}_2   \\
  \dot{\eta}_3   \\  %these are dotted lines, yea, that's LaTeX for ya, go figure..
  %\begin{picture} (0,0)(0,0) \multiput(-2,14)(4,0){3}{\line(2,0){2}} \end{picture}
  \dot{\xi}
  \end{bmatrix} 
  &=
  \overbrace{
    \underbrace{
      \begin{bmatrix}
      \frac{\cos \eta_1}{l} \xi - \eta_2    \\
      %\frac{l \sin x1}{\cos^2 x_1} x_3^2 +  \tfrac{l}{\cos x_1} f_1(\vec{x})  - f_2(\vec{x}) \\ 
      \frac{-\sin \eta_1}{l} (\frac{\cos \eta_1}{l} \xi - \eta_2) \xi + \frac{\cos \eta_1}{l} f_4(\vec{\eta},\xi) - f_3(\vec{\eta},\xi) \\
      \xi    \\ %these are dotted lines, yea, that's LaTeX for ya, go figure..
      \begin{picture} (0,0)(0,0) \multiput(-95,14)(4,0){57}{\line(2,0){2}} \end{picture}
      f_4(\vec{\eta},\xi) 
      \end{bmatrix}
    }_{f_b} }^{f_a}
  +
  \underbrace{
    \begin{bmatrix}
    0    \\
    0    \\
    0    \\  %these are dotted lines, yea, that's LaTeX for ya, go figure..
    \begin{picture} (0,0)(0,0) \multiput(0,14)(4,0){10}{\line(2,0){2}} \end{picture}
    g_4(\vec{\eta},\xi)
    \end{bmatrix}
  }_{g_b} u  \ \ \ ,
  \label{eq:regularFormMatrix}
\end{align}
where,
\begingroup\makeatletter\def\f@size{10}\check@mathfonts
\def\maketag@@@#1{\hbox{\m@th\normalsize\normalfont#1}}%
\begin{align}
f_3(\vec{\eta},\xi) &= \frac{1}{ l^2 m (M + m - m \cos^2 \eta_1) }
\left[
(M + m) b_{p,v} \left(\eta_2 - \frac{\cos \eta_1 \xi}{l}\right) + \right. \nonumber \\
&+ (M + m) b_{p,c} \tanh\left(\mathrm{k}_{\mathrm{tanh}} \left(\eta_2 - \frac{ \cos \eta_1 \xi}{l}\right)\right) + m^2 g l \sin \eta_1 - b_{c,c} m l \tanh\left(\mathrm{k}_{\mathrm{tanh}} \xi\right) \cos \eta_1 - \nonumber \\
&- \left. m^2 l^2 \cos \eta_1 \sin \eta_1 \left(\eta_2 - \frac{\xi \cos \eta_1}{l}\right)^2 + M g l m \sin \eta_1 - b_{c,v} m l \xi \cos \eta_1
\right] \\
%
%
f_4(\vec{\eta},\xi) &= -\frac{1}{ l (M + m - m \cos^2 \eta_1) } \left[   b_{c,v} l \xi - b_{p,v} \cos \eta_1 \left(\eta_2 - \frac{\cos \eta_1 \xi}{l}\right) + b_{c,c} l \tanh\left(\mathrm{k}_{\mathrm{tanh}} \xi\right) - \right. \nonumber \\
&- b_{p,c} \tanh\left(\mathrm{k}_{\mathrm{tanh}} \left(\eta_2 - \frac{ \cos \eta_1 \xi}{l}\right)\right) \cos \eta_1 + \nonumber \\
&+ \left. l^2 m \sin \eta_1 \left(\eta_2 - \frac{\xi \cos \eta_1}{l}\right)^2 - m g l \cos \eta_1 \sin \eta_1  \right] \\
%
%
g_4(\vec{\eta},\xi) &= \frac{1}{M + m - m \cos^2 \eta_1 }
\ \ \ .
\label{eq:f3f4g4} \\ \nonumber
\end{align}
\endgroup \vspace{-44pt}

%
%
%
%f3 = \frac{ M b_{p,v} (\eta_2 - \frac{\xi \cos \eta_1}{l}) + b_{p,v} m (\eta_2 - \frac{\xi \cos \eta_1}{l}) + M b_{p,c} \tanh(k_{\mathrm{tanh}} (\eta_2 - \frac{\xi \cos \eta_1}{l})) + b_{p,c} m \tanh(k_{\mathrm{tanh}} (\eta_2 - \frac{\xi \cos \eta_1}{l})) + g l m^2 \sin \eta_1 - b_{c,c} l m \tanh(k_{\mathrm{tanh}} \xi) \cos \eta_1 - l^2 m^2 \cos \eta_1 \sin \eta_1 (\eta_2 - \frac{\xi \cos \eta_1}{l})^2 + M g l m \sin \eta_1 - b_{c,v} l m \xi \cos \eta_1 }{ l^2 m (M + m - m \cos^2 \eta_1) }
%
%
%
%f4 = -\frac{1}{ l (M + m - m \cos^2 \eta_1) } \left[   b_{c,v} l \xi - b_{p,v} \cos \eta_1 \left(\eta_2 - \frac{\cos \eta_1 \xi}{l}\right) + b_{c,c} l \tanh\left(\mathrm{k}_{\mathrm{tanh}} \xi\right) - b_{p,c} \tanh\left(\mathrm{k}_{\mathrm{tanh}} \left(\eta_2 - \frac{ \cos \eta_1 \xi}{l}\right)\right) \cos \eta_1 + l^2 m \sin \eta_1 \left(\eta_2 - \frac{\xi \cos \eta_1}{l}\right)^2 - m g l \cos \eta_1 \sin \eta_1  \right]
%
%
\fxnote{Is this too much?}


%\begin{align}
%s &=   \xi - \phi(\vec{\eta}) = 0     \ \ \ ,
%\end{align}
%design $\phi(\vec{\eta})$
%
%if $s = 0$ then $\xi = \phi(\vec{\eta})$, that is, the reduced-order system,
%\begin{align}
%  \vec{\dot{\eta}} &=  f_a(\vec{\eta},\phi(\vec{\eta}))     \ \ \ ,
%  \label{eq:asymStabOrigReducedOrder}
%\end{align}
%is asymptotically stable in the origin.
%
%linearization of reduced-order system,
%%
%\begin{align}
%A &= \frac{\partial \vec{\dot{\eta}}}{\partial \vec{\eta}} \whereThree{\vec{\eta}=\vec{0}\ \ \ \ }{\xi=0\ \ \ \ }{\text{k}_\text{tanh}=1} \ 
%=
%\begin{bmatrix}
%0 & -1                  & 0 \\
%0 & \frac{g_{p,c}}{l m} & g \\
%0 & 0                   & 0 
%\end{bmatrix}   \ \ \ , \ \ \
%B = \frac{\partial \vec{\dot{\eta}}}{\partial \xi} \whereThree{\vec{\eta}=\vec{0}\ \ \ \ }{\xi=0\ \ \ \ }{\text{k}_\text{tanh}=1} \ 
%=
%\begin{bmatrix}
%l  \\
%\frac{-b_{p,v}-b_{p,c}}{l m}  \\
%1  
%\end{bmatrix}   \ \ \ ,
%\label{eq:linearReducedOrder_A}
%\end{align}
%
%stabilize reduced order system by,
%\begin{align}
%\phi(\vec{\eta}) &=   - \vec{k} \vec{\eta}  \ \ \ ,
%\end{align}
%
%design $u$ to bring $s$ to zero.\\
%Lyapunov function candidate, $V = \frac{1}{2}s^2$,
%derivative along trajectories of system
%\begin{align}
%\dot{V} &= s\dot{s} \\
%\dot{V} &= s ( \dot{\xi} + \vec{k}\vec{\dot{\eta}}  ) \\
%\dot{V} &= s ( f_b(\vec{\eta},\xi) + g_b(\vec{\eta},\xi) u +\vec{k}f_a(\vec{\eta},\xi) )  \\
%\dot{V} &= (\vec{k}f_a +  f_b) s + g_b s u \\
%\dot{V} &= g_b s (\vec{k}f_a  +  f_b) g_b^{-1} + g_b s u   \\
%\dot{V} &\leq g_b |s| \left|\vec{k}f_a g_b^{-1} +  f_b \right| + g_b s u  \ \ \ .
%\label{eq:lyapunov}
%\end{align}
%
%using $|s| = \text{sgn}(s) s$ and introducing tuning parameter $\beta_0$ to design $u$,
%\begin{align}
%u = -\text{sgn}(s)\beta (\vec{\eta},\xi) g_b^{-1}(\vec{\eta},\xi) \ \ \ \ \mathrm{where}, \ \ \ \beta(\vec{\eta},\xi)  &= \varrho(\vec{\eta},\xi) + \beta_0 \\
%\mathrm{and}, \ \ \  \varrho(\vec{\eta},\xi) &= \left|\vec{k}f_a  +  f_b \right| \ \ \ .
%\label{eq:ssControlBeta0}
%\end{align}
%s.t.
%\begin{align}
%\dot{V} &\leq g_b |s| \left|\vec{k}f_a +  f_b \right|  g_b^{-1} - g_b\  \mathrm{sgn}(s) s \left|\vec{k}f_a  +  f_b + \beta_0 \right| g_b^{-1}
%\label{eq:lyapunov2}
%\end{align}
%
%
%sign-function discontinuity\\
%sat-function used
%\begin{align}
%\text{sat}\left( \tfrac{s}{\varepsilon} \right) &=
%\begin{cases}
%\ \ \frac{s}{\varepsilon},                           & \ \text{if} \ | \frac{s}{\varepsilon} | \leq 1 \\
%\ \ \text{sgn}\left( \tfrac{s}{\varepsilon} \right), & \ \text{if} \ | \frac{s}{\varepsilon} |  >   1 \ \ \ ,
%\end{cases}
%\label{eq:satuationFunction2}
%\end{align}
%where $\frac{1}{\varepsilon}$ slope
%\begin{align}
%u &= -\text{sat}(\tfrac{s}{\varepsilon})\beta (\vec{\eta},\xi)  g_b^{-1}(\vec{\eta},\xi)  \ \ \ .
%\label{eq:ssControlSat}
%\end{align}
%%

